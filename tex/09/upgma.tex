\question \textbf{UPGMA}

UPGMA is an unweighted version of PGMA (pair-group method using arithmetic mean) for reconstructing a phylogenetic tree. Pairwise sequence alignments are used to calculate the distances among four sequences A, B, C, and D.

\begin{table}[H]
\centering
\begin{tabular}{lllll}
                       & A                      & B                      & C                      & D                      \\ \cline{2-5} 
\multicolumn{1}{l|}{A} & \multicolumn{1}{l|}{0} & \multicolumn{1}{l|}{2} & \multicolumn{1}{l|}{7} & \multicolumn{1}{l|}{7} \\ \cline{2-5} 
B                      & \multicolumn{1}{l|}{}  & \multicolumn{1}{l|}{0} & \multicolumn{1}{l|}{5} & \multicolumn{1}{l|}{9} \\ \cline{3-5} 
C                      &                        & \multicolumn{1}{l|}{}  & \multicolumn{1}{l|}{0} & \multicolumn{1}{l|}{8} \\ \cline{4-5} 
D                      &                        &                        & \multicolumn{1}{l|}{}  & \multicolumn{1}{l|}{0} \\ \cline{5-5} 
\end{tabular}
\end{table}

Below are two examples of the distance calculation that can be used for UPGMA.

\[
d_{(\alpha\beta),\gamma}=\dfrac{d_{\alpha,\gamma} + d_{\beta,\gamma}}{2}
\]

\[
d_{(\alpha\beta\gamma),\delta}=\dfrac{d_{\alpha,\delta} + d_{\beta,\gamma} + d_{\delta,\gamma}}{3}
\]

\begin{parts}

\vspace{0.1 in}

%% (a)
  \part Identify the first internal node and update the distance matrix accordingly.
  
\begin{table}[H]
\centering
\begin{tabular}{cccc}
{\color[HTML]{FFFFFF} }                          & {\color[HTML]{FFFFFF} (AB)}                   & {\color[HTML]{FFFFFF} C}                      & {\color[HTML]{FFFFFF} D}                      \\ \cline{2-4} 
\multicolumn{1}{c|}{{\color[HTML]{FFFFFF} (AB)}} & \multicolumn{1}{c|}{{\color[HTML]{FFFFFF} 0}} & \multicolumn{1}{c|}{{\color[HTML]{FFFFFF} 6}} & \multicolumn{1}{c|}{{\color[HTML]{FFFFFF} 8}} \\ \cline{2-4} 
{\color[HTML]{FFFFFF} C}                         & \multicolumn{1}{c|}{{\color[HTML]{FFFFFF} }}  & \multicolumn{1}{c|}{{\color[HTML]{FFFFFF} 0}} & \multicolumn{1}{c|}{{\color[HTML]{FFFFFF} 8}} \\ \cline{3-4} 
{\color[HTML]{FFFFFF} D}                         & {\color[HTML]{FFFFFF} }                       & \multicolumn{1}{c|}{{\color[HTML]{FFFFFF} }}  & \multicolumn{1}{c|}{{\color[HTML]{FFFFFF} 0}} \\ \cline{4-4} 
\end{tabular}
\end{table}

%% (b)
  \part Identify the second internal node and update the distance matrix accordingly.
  
  \begin{table}[H]
\centering
\begin{tabular}{ccc}
{\color[HTML]{FFFFFF} }                           & {\color[HTML]{FFFFFF} (ABC)}                 & {\color[HTML]{FFFFFF} D}                     \\ \cline{2-3} 
\multicolumn{1}{c|}{{\color[HTML]{FFFFFF} (ABC)}} & \multicolumn{1}{c|}{{\color[HTML]{FFFFFF} }} & \multicolumn{1}{c|}{{\color[HTML]{FFFFFF} }} \\ \cline{2-3} 
{\color[HTML]{FFFFFF} D}                          & \multicolumn{1}{c|}{{\color[HTML]{FFFFFF} }} & \multicolumn{1}{c|}{{\color[HTML]{FFFFFF} }} \\ \cline{3-3} 
\end{tabular}
\end{table}

%% (c)
  \part Reconstrut a rooted tree from the calcualted distances.
  
  \vspace{1.4 in}
  
%% (d)
  \part  Fill the distances of the reconstructed tree.
  
\begin{table}[H]
\centering
\begin{tabular}{lllll}
                       & A                      & B                      & C                      & D                      \\ \cline{2-5} 
\multicolumn{1}{l|}{A} & \multicolumn{1}{l|}{0} & \multicolumn{1}{l|}{} & \multicolumn{1}{l|}{} & \multicolumn{1}{l|}{} \\ \cline{2-5} 
B                      & \multicolumn{1}{l|}{}  & \multicolumn{1}{l|}{0} & \multicolumn{1}{l|}{} & \multicolumn{1}{l|}{} \\ \cline{3-5} 
C                      &                        & \multicolumn{1}{l|}{}  & \multicolumn{1}{l|}{0} & \multicolumn{1}{l|}{} \\ \cline{4-5} 
D                      &                        &                        & \multicolumn{1}{l|}{}  & \multicolumn{1}{l|}{0} \\ \cline{5-5} 
\end{tabular}
\end{table}
  
\end{parts}

