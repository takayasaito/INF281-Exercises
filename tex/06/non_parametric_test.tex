\question \textbf{Non-parametric test}
  
A non-parametric test is used to determine a p-value for the optimal score of a global alignment. Assume we randomly generated 9 sequences and calculated the alignment scores as follows.

\begin{verbatim}
    q: AACG
\end{verbatim}

\begin{table}[H]
\centering
\begin{tabular}{|l|l|l|l|l|l|l|l|l|l|}
\hline
Seq No. & 1   & 2   & 3   & 4   & 5   & 6   & 7   & 8   & 9   \\ \hline
Score   & 0.2 & 0.4 & 0.5 & 1.2 & 1.2 & 1.5 & 1.9 & 2.2 & 2.1 \\ \hline
\end{tabular}
\end{table}

\vspace{0.1 in}

\begin{parts}

%% (a)
  \part What are $H_{0}$ (null hypothesis) and $H_{1}$ (alternative hypothesis) if you want to use a statistical hypothesis test to evaluate a global pairwise alignment in terms of finding homologues?

\begin{solution}[0.35 in]
$H_{0}$:  Sequences are not homologous, $H_{1}$:  Sequences are homologous
\end{solution}

%% (b)
\part Calculate the p-value for the alignment below. 

\begin{verbatim}
    q: AACG
    d: AGTG

    Score: 2
\end{verbatim}
\medskip 

The p-value can be calculated as:

\begin{center}
$p=(b+1)/(n+1)$
\end{center}

where $b$ is the number of randomly generated scores above the score of the original alignment, and $n$ is the sample size. 

\begin{solution}[0.35 in]
p-value:  0.3
\end{solution}

%% (c)
\part Is the test result statistically significant when $\alpha$ = 0.05?

\begin{solution}[0.35 in]
No.
\end{solution}

%% (d)
\part What is the conclusion of the test in terms of finding homologues?

\begin{solution}[0.35 in]
q and d are not homologous.
\end{solution}

\end{parts}

