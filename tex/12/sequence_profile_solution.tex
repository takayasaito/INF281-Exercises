\question \textbf{Sequence profile}

A sequence profile is similar to PWM, but it uses a scoring scheme. Use the following definitions to calculate the profile values.
\begin{align*}
Prof_{ra} &: \dfrac{1}{m_r} \sum_{b \in M} R_{ba}F_{rb} \\
F_{rb} &: \text{The number of occurrences of } b \text{ at position } r \\
R_{ba} &: \text{Pairwise score between } b \text{ and } a \\
m_r &: \text{The number of residues without gaps at position } r
\end{align*}

Scoring matrix:
\begin{table}[H]
\centering
\begin{tabular}{|c|c|c|c|c|}
\hline
  & A & G & C & T \\ \hline
A &  2 &  1 & -3  & -2  \\ \hline
G &  1 &  3 &  -2 & -1  \\ \hline
C & -3 & -2  & 4  & 1  \\ \hline
T & -2 & -1  & 1  &  2 \\ \hline
\end{tabular}
\end{table}

MSA
\begin{verbatim}
   Seq1 GT
   Seq2 -G
   Seq3 CA
\end{verbatim}

\vspace{0.1 in}

\begin{parts}

%% (a)
  \part Calculate the profile values of position 1.

A1: \colorbox{SolutionColor}{$(1/2) \times ((2 \times 0) + (1\times 1) + (-3 \times 1)+ (-2 \times 0)) = -1$}  \\
G1: \colorbox{SolutionColor}{$(1/2) \times ((1 \times 0) + (3 \times 1) + (-2 \times 1)+ (-1 \times 0)) = 1/2$}  \\
C1: \colorbox{SolutionColor}{$(1/2) \times ((-3 \times 0) + (-2 \times 1) + (4 \times 1)+ (1 \times 0)) = 1$}   \\
T1: \colorbox{SolutionColor}{$(1/2) \times ((-2 \times 0) + (-1 \times 1) + (1 \times 1)+ (2 \times0)) = 0$}   \\

%% (b)
  \part Calculate the profile values of position 2.

A2: \colorbox{SolutionColor}{$(1/3) \times ((2 \times 1) + (1 \times 1) + (-3 \times 0)+ (-2 \times 1)) = 1/3$}  \\
G2: \colorbox{SolutionColor}{$(1/3)  \times ((1 \times 1) + (3 \times 1) + (-2 \times 0)+ (-1 \times 1)) = 1$}  \\
C2: \colorbox{SolutionColor}{$(1/3)  \times ((-3 \times 1) + (-2 \times 1) + (4 \times 0)+ (1 \times 1)) = -4/3$}  \\
T2: \colorbox{SolutionColor}{$(1/3)  \times ((-2 \times 1) + (-1 \times 1) + (1 \times 0)+ (2 \times 1)) = -1/3$}  \\

%% (c)
  \part Make a profile matrix.
  
\begin{table}[H]
\centering
\begin{tabular}{|c|c|c|}
\hline
  & 1                        & 2                        \\ \hline
A & \cellcolor[HTML]{CCE5FF} -1 & \cellcolor[HTML]{CCE5FF}  1/3 \\ \hline
G & \cellcolor[HTML]{CCE5FF} 1/2 & \cellcolor[HTML]{CCE5FF} 1 \\ \hline
C & \cellcolor[HTML]{CCE5FF} 1 & \cellcolor[HTML]{CCE5FF} -4/3 \\ \hline
T & \cellcolor[HTML]{CCE5FF} 0 & \cellcolor[HTML]{CCE5FF} -1/3 \\ \hline
\end{tabular}
\end{table}

\end{parts}
