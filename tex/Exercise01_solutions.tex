% See exam.cls and examdoc.tex for the license information
\documentclass[12pt, answers]{exam}

\usepackage{amssymb}
\usepackage{makeidx}
\usepackage{amsmath}
\usepackage{graphicx}
\usepackage{caption}
\usepackage{tabulary}
\usepackage{color}
\usepackage{multicol}
%\usepackage{multirow}
%\usepackage{enumerate}

\usepackage{array}
\newcolumntype{C}[1]{>{\centering\let\newline\\\arraybackslash\hspace{0pt}}m{#1}}

\addpoints

% In case we're not using hyperref.sty:
\providecommand{\texorpdfstring}[2]{#1}
% The following can be used in \section commands
% without generating pdf warnings:
\newcommand{\bs}{\texorpdfstring{\char`\\}{}}

\makeindex

\newcommand{\indc}[1]{\index{#1@\texttt{\char`\\#1}}}
\newcommand{\indcsub}[2]{\index{#1@\texttt{\char`\\#1}!#2}}
\newcommand{\indcstart}[1]{\index{#1@\texttt{\char`\\#1}|(}}
\newcommand{\indcstop}[1]{\index{#1@\texttt{\char`\\#1}|)}}

\newcommand{\indt}[1]{\index{#1@\texttt{#1}}}
\newcommand{\indtsub}[2]{\index{#1@\texttt{#1}!#2}}
\newcommand{\indtstart}[1]{\index{#1@\texttt{#1}|(}}
\newcommand{\indtstop}[1]{\index{#1@\texttt{#1}|)}}

\extraheadheight{-.4in}

\pagestyle{headandfoot}
%\extraheadheight{.2 in}
\firstpageheader{}{}{}
\runningheader{}{}{}
\firstpagefooter{INF281}{Exercise 01}{Page \thepage\ of \numpages}
\firstpagefootrule
\runningfooter{INF281}{Exercise 01}{Page \thepage\ of \numpages}
\runningfootrule

%---------------------------------------------------------------------

\shadedsolutions
%\noprintanswers
\definecolor{SolutionColor}{rgb}{0.8,0.9,1}

\begin{document}

\section*{INF281 Exercise 01}

%---------------------------------------------------------------------
\begin{questions}

%%% Question 1
\question \textbf{Eukaryotic and prokaryotic cells}
 
 Cells are the most basic building blocks of all living organisms. Several aspects differ between prokaryotic and eukaryotic cells.
  
\vspace{0.1 in}
      
\begin{parts}
  
%% 1-(a)
  \part
  Which one of the two cell types, prokaryotes or eukaryotes, lacks a distinct nucleus?
  
  \begin{solution}[0.35 in]
  Prokaryotes
  \end{solution}
  
%% 1-(b)
  \part
  Can eukaryotes be unicellular?
  
  \begin{solution}[0.35 in]
  Yes
  \end{solution}

  \vspace{0.1 in}

\end{parts}

%%% Question 2
\question \textbf{Cell organelles}
  
A cell consists of a number of specialized subunits called organelles. Answer the most suitable organell that matchs the explanation. 

\vspace{0.1 in}

\begin{parts}

%% 2-(a)
  \part
  The key role of this organelle is to produce energy rich molecules for the cell.
  
  \begin{solution}[0.35 in]
  Mitochondria
  \end{solution}

%% 2-(b)
  \part
  It is a membrane-enclosed organelle. It contains genetic material called chromosomes.
  
  \begin{solution}[0.35 in]
  Nucleus
  \end{solution}
  
  \vspace{0.1 in}

\end{parts}
 
%%% Question 3
\question \textbf{DNA}
  
Deoxyribonucleic acid (DNA) is an important molecule that stores generic information. DNA consists of four different nucleotides - Adenine (A), Cytosine (C), Guanine (G), and Thymine (T).

\vspace{0.1 in}

\begin{parts}

%% 3-(a)
  \part
  What is the DNA sequence when its opposite strand is ACCGT?
  
  \begin{solution}[0.35 in]
  TGGCA
  \end{solution}
  
  \vspace{0.1 in}

\end{parts}

%%% Question 4
\question \textbf{Central dogma of molecular biology}
  
The central dogma of molecular biology describes the flow of genetic information by three processes - Replication, Transcription, and Translation.

\vspace{0.1 in}

\begin{parts}

%% 4-(a)
  \part
  What are the two main molecules involved in transcription?
  
  \begin{solution}[0.35 in]
  DNA and RNA
  \end{solution}

%% 4-(b)
  \part
  What are the two main molecules involved in translation?
  
  \begin{solution}[0.35 in]
  RNA and protein
  \end{solution}
  
  \vspace{0.1 in}

\end{parts}

%%% Question 5
\question \textbf{Central dogma of molecular biology}
  
Ribonucleic acid (RNA) conveys genetic information from its corresponding DNA to ribosome where proteins are synthesized. Similar to DNA, RNA has four different forms - Adenine (A), Cytosine (C), Guanine (G), and Uracil (U).

\vspace{0.1 in}

\begin{parts}

%% 5-(a)
  \part
  What is the transcribed RNA sequence when the corresponding DNA sequence is TATAGC?
  
  \begin{solution}[0.35 in]
  UAUAGC
  \end{solution}
  
  \vspace{0.1 in}

\end{parts}

%%% Question 6
\question \textbf{Genetic code}
  
The genetic code defines how three RNA nucleotides, called codon, should be translated into an amino-acid. 

\begin{table}[h]
\footnotesize
\begin{center}
\begin{tabular}{|c|cccc|c|}
\hline
\itshape First  & \multicolumn{4}{c|}{\itshape Second position} & \itshape Third \\
\cline{2-5}
\itshape position & \makebox[2em]{T} & \makebox[2em]{C} & \makebox[2em]{A} & \makebox[2em]{G} & \itshape position \\ 
\hline
   & F & S & Y & C & T \\
   & F & S & Y & C & C \\
\raisebox{1.5ex}[0mm]{T} & L & S & \itshape Stop & \itshape Stop & A \\
   & L & S & \itshape Stop & W & G \\ 
\hline
   & L & P & H & R & T \\
   & L & P & H & R & C \\
\raisebox{1.5ex}[0mm]{C}    & L & P & Q & R & A \\
   & L & P & Q & R & G \\ 
\hline
   & I & T & N & S & T \\
   & I & T & N & S & C \\
\raisebox{1.5ex}[0mm]{A}    & I & T & K & R & A \\
   & M & T & K & R & G \\ 
\hline
   & V & A & D & G & T \\
   & V & A & D & G & C \\
\raisebox{1.5ex}[0mm]{G}    & V & A & E & G & A \\
   & V & A & E & G & G \\ 
\hline
\end{tabular}
%\epsfig{file=gencode.eps,width=.25\textwidth}
\qquad
\begin{tabular}{lll}
A & Ala & Alanine \\
C & Cys & Cysteine \\
D & Asp & Aspartic acid \\
E & Glu & Glutamic acid \\
F & Phe & Phenylalanine \\
G & Gly & Glycine \\
H & His & Histidine \\
I & Ile & Isoleucine \\
K & Lys & Lysine \\
L & Leu & Leucine \\
M & Met & Methionine \\
N & Asn & Asparagine \\
P & Pro & Proline \\
Q & Gln & Glutamine \\
R & Arg & Arginine \\
S & Ser & Serine \\
T & Thr & Threonine \\
V & Val & Valine \\
W & Trp & Tryptophan \\
Y & Tyr & Tyrosine
\end{tabular}
\end{center}
\end{table}

\begin{parts}

%% 6-(a)
  \part
  Stop codons are special codons that terminate the protein synthesis. Specify all three stop codons.
  
  \begin{solution}[0.35 in]
  UAA, UGA, UAG
  \end{solution}

%% 6-(b)
  \part
  Both one-letter and three-letter abbreviations are often used instead of the full names. What is the three-letter abbreviation of Tyrosine?
  
  \begin{solution}[0.35 in]
  Tyr
  \end{solution}
  
%% 6-(c)
  \part
  What is the one-letter abbreviation of the amino acid that is synthesized from the codon AGA?   
  
  \begin{solution}[0.35 in]
  R (Arginine)
  \end{solution}
    
  \vspace{0.1 in}

\end{parts}

%%% Question 7
\question \textbf{Pairwise alignments}
  
Pairwise alignments are two aligned sequences of DNA, RNA, or protein. DNA and RNA sequences consist of four different nucleotides, whereas protein sequences consist of 20 different amino acids. The “-” sign is used to represent a blank or a gap, which indicates an insertion or a deletion from one sequence to the other.

Use the simple scoring scheme and calcuate the score of the following alignments.

\textbf{Scoring scheme: }\\
\null \quad $R_{ab}$ = 1 for a = b \\ 
\null \quad $R_{ab}$ = 0 for a $\neq$ b \\ 
\null \quad g = 1  

\vspace{0.1 in}

\begin{parts}

%% 7-(a)
  \part
  Alignment 1
  \begin{verbatim}
    q: ATGCT
    d: CA--T \end{verbatim}
    
  \begin{solution}[0.35 in]
  -1
  \end{solution}

%% 7-(b)
  \part
  Alignment 2
  \begin{verbatim}
  q: CAGCT
  d: C-A-T \end{verbatim}
  
  \begin{solution}[0.35 in]
  0
  \end{solution}
    
  \vspace{0.1 in}

\end{parts}

%%% Question 8
\question \textbf{Brute force approach}
  
A brute force approach can be used to find the optimal alignment. Use the sequences $q$ and $d$ below and answer the questions.

\begin{multicols}{2}
Sequences:
\begin{verbatim}
  q: CG, d: AC
\end{verbatim}
\vfill\null
\columnbreak

\noindent Scoring scheme: \\ 
\null \quad $R_{ab}$ = 1 for a = b \\ 
\null \quad $R_{ab}$ = 0 for a $\neq$ b \\ 
\null \quad g = 1

\end{multicols} 

\vspace{0.1 in}

\begin{parts}

%% 8-(a)
  \part
  Identify all possible alignments.
  
  \begin{solution}[2.5 in]
  \begin{verbatim}
  l = 4  CG--    C-G-    C--G    -CG-    -C-G    --CG
         --AC    -A-C    -AC-    A--C    A-C-    AC--

  l = 3  CG-   CG-   C-G    C-G    -CG    -CG
         A-C   -AC   AC-    -AC    A-C    AC-

  l = 2  CG
         AC	
  \end{verbatim}
  \end{solution}

%% 8-(b)
  \part
  Identify the optimal alignment with its score.
  
  \begin{solution}[0.75 in]
  \begin{verbatim}
  CG
  AC    Score: 0
  \end{verbatim}

  \end{solution}
    
  \vspace{0.1 in}

\end{parts}

%%% Question 9
\question \textbf{Brute force approach}
  
An alignment can be represented as a table with arrows. Vertical and horizontal arrows indicate gaps, while diagonal arrows indicate matches and mismatches.

Identify the alignment that corresponds to the arrows in the following tables.

\begin{parts}

%% 9-(a)
\part Table 1
\begin{figure}[h]
  \centering
      \includegraphics[width=0.35 \textwidth]{fig01/alignment_table_exercise_1.png}
\end{figure}

\begin{solution}[0.75 in]
\begin{verbatim}
  q: CAGCT
  d: CA--T
\end{verbatim}
\end{solution}
  
%% 9-(b)
\part Table 2
\begin{figure}[h]
  \centering
      \includegraphics[width=0.35 \textwidth]{fig01/alignment_table_exercise_2.png}
\end{figure}

\begin{solution}[0.75 in]
\begin{verbatim}
  q: -CAGCT-
  d: C----AT
\end{verbatim}
\end{solution}

\end{parts}

\end{questions}
%---------------------------------------------------------------------
       
\end{document}

